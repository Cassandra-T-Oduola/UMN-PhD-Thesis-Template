% Physics constants
\newcommand{\C}{{\mathrm{c}}}

% Energy
% We move the V slightly closer to make it look better
\newcommand{\EnergyIneV}[1]{\ensuremath{\,\text{#1e\hspace{-.08em}V}}\xspace}
\newcommand{\eV}{\EnergyIneV{}}
\newcommand{\keV}{\EnergyIneV{k}}
\newcommand{\MeV}{\EnergyIneV{M}}
\newcommand{\GeV}{\EnergyIneV{G}}
\newcommand{\TeV}{\EnergyIneV{T}}

% Momentum
% We define the momentum (and mass) as \EnergyIneV / c^n where n is taken as an
% argument. This way any changes to formatting change all three equally.
\newcommand{\EnergyWithCPower}[2]{\ensuremath{{\EnergyIneV{#1}\text{\hspace{-0.16em}/\hspace{-0.08em}}c^{\text{#2}}}}\xspace}
\newcommand{\MomentumIneVc}[1]{\EnergyWithCPower{#1}{}}
\newcommand{\eVc}{\MomentumIneVc{}}
\newcommand{\keVc}{\MomentumIneVc{k}}
\newcommand{\MeVc}{\MomentumIneVc{M}}
\newcommand{\GeVc}{\MomentumIneVc{G}}
\newcommand{\TeVc}{\MomentumIneVc{T}}

% Mass
\newcommand{\MassIneVcc}[1]{\EnergyWithCPower{#1}{2}}
\newcommand{\eVcc}{\MassIneVcc{}}
\newcommand{\keVcc}{\MassIneVcc{k}}
\newcommand{\MeVcc}{\MassIneVcc{M}}
\newcommand{\GeVcc}{\MassIneVcc{G}}
\newcommand{\TeVcc}{\MassIneVcc{T}}

% Add space between rows of tables
\newcommand{\spacerows}[1]{\renewcommand{\arraystretch}{#1}}
